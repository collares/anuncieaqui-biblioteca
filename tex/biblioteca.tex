\documentclass[a4paper]{amsart}
\usepackage[pdftex]{geometry}
\usepackage[utf8]{inputenc}

\usepackage{listings}
\usepackage{amssymb}
\usepackage{amsmath}
\usepackage{multicol}
\usepackage{courier}
\usepackage{fancyhdr}

\lstset{language=C++}
\lstset{columns=fullflexible}
\lstset{basicstyle=\scriptsize\ttfamily}
\lstset{showspaces=false}

\geometry{landscape}
\setlength{\topmargin}{-0.25in}
\setlength{\oddsidemargin}{0in}
\setlength{\evensidemargin}{0in}
\setlength{\columnsep}{1in}
\setlength{\columnseprule}{0.2pt}
\textwidth 9.5in

\pagestyle{fancy}
\lhead{Universidade Federal de Sergipe}
\chead{}
\rhead{\thepage}
\lfoot{}
\cfoot{}
\rfoot{}

\newcommand{\includefile}[3]{
  \begin{multicols}{2}
    [\subsection{#1}{Hash: \input{#2/.#3.hash}}\ ]
    \lstinputlisting{#2/#3}
  \end{multicols}
}

\newcommand{\stirlingfirst}[2]{\genfrac{[}{]}{0pt}{}{#1}{#2}}
\newcommand{\stirlingsecond}[2]{\genfrac{\{}{\}}{0pt}{}{#1}{#2}}
\newenvironment{tablehere}{\def\@captype{table}}{}

\begin{document}
  \thispagestyle{fancy}
  \begin{center}
    \Huge\textsc{ACM ICPC Team Reference \\ 2010 World Finals}

    \ 

    \huge Team Anuncie Aqui \\ Universidade Federal de Sergipe
 
    \ 

  \end{center}

%  \begin{multicols}{2}
%    \tableofcontents
%  \end{multicols}

  \section{Configuration files and scripts}
    \includefile{.emacs}{../config}{dotemacs}
    \includefile{.vimrc}{../config}{dotvimrc}
    \includefile{Hash generator}{../config}{hashgen}
    \includefile{Solution template}{../config}{template.cpp}

  \section{Graph algorithms}
    \includefile{Tarjan's SCC algorithm}{../graph}{scc.cpp}
    \includefile{Dinic's algorithm}{../graph}{dinic.cpp}
    \includefile{Busacker-Gowen's algorithm}{../graph}{busacker_gowen.cpp}
    \includefile{Gabow's algorithm}{../graph}{gabow.cpp}

  \section{Math}
    \includefile{Fractions}{../math}{frac.cpp}
    \includefile{Chinese remainder theorem}{../math}{crt.cpp}
    \includefile{Longest increasing subsequence}{../math}{lis.cpp}
    \includefile{Simplex (Warsaw University)}{../math}{simplex.cpp}
    \includefile{Romberg's method}{../math}{romberg.cpp}
    \includefile{Floyd's cycle detection algorithm}{../math}{floyd.cpp}
    \includefile{Pollard's rho algorithm}{../math}{pollard.cpp}
    \includefile{Miller-Rabin's algorithm}{../math}{miller_rabin.cpp}
    \includefile{Polynomials (PUC-Rio)}{../math}{polynomials.cpp}

  \section{Geometry}
    \includefile{Point class}{../geometry}{point.cpp}
    \includefile{Intersection primitives}{../geometry}{isect_primitives.cpp}
    \includefile{Polygon primitives}{../geometry}{polygon_primitives.cpp}
    \includefile{Miscellaneous primitives}{../geometry}{misc_primitives.cpp}
    \includefile{Smallest enclosing circle}{../geometry}{enclosing_circle.cpp}
    \includefile{Convex hull}{../geometry}{hull.cpp}
    \includefile{Closest pair of points}{../geometry}{closest_points.cpp}
    \includefile{Kd-tree}{../geometry}{kd_tree.cpp}
    \includefile{Range tree}{../geometry}{range_tree.cpp}

  \section{Data structures}
    \includefile{Treap}{../structures}{treap.cpp}
    \includefile{Heap}{../structures}{heap.cpp}
    \includefile{Big numbers (PUC-Rio)}{../structures}{bignum.cpp}

  \section{String algorithms}
    \includefile{Manber-Myers' algorithm}{../string}{manber_myers.cpp}
    \includefile{Morris-Pratt's algorithm}{../string}{morris_pratt.cpp}

  \enlargethispage*{\baselineskip}
  \pagebreak

  \section{Useful mathematical facts}
  \begin{multicols}{2}
    \subsection{Prime counting function ($\pi(x)$)} The prime counting function is asymptotic to $\frac{x}{\log x}$, by the prime number theorem.

      \ 

      \begin{tabular}{|c|c|c|c|c|c|c|c|c|}
        \hline 
                  & 10 & $10^2$ & $10^3$ & $10^4$ & $10^5$ & $10^6$ & $10^7$ & $10^8$\\ \hline
          $\pi(x)$&  4 &   25 & 168 & 1.229 & 9.592 & 78.498 & 664.579 & 50.847.534\\ \hline
      \end{tabular}

    \subsection{Catalan numbers} Catalan numbers are defined by the recurrence:
      \begin{equation*}
        C_{n+1} = \sum_{i=0}^nC_iC_{n-i}
      \end{equation*}

      A closed formula for Catalan numbers is:
      \begin{equation*}
        C_n = \frac{1}{n+1}\binom{2n}{n} = \binom{2n}{n} - \binom{2n}{n+1}
      \end{equation*}

    \subsection{Stirling numbers of the first kind} These are the number of permutations of $I_n$ with exactly $k$ disjoint cycles. They obey the recurrence:
      \begin{equation*}
        \stirlingfirst{n}{k} = (n-1)\stirlingfirst{n-1}{k} + \stirlingfirst{n-1}{k-1}
      \end{equation*}

    \subsection{Stirling numbers of the second kind} These are the number of ways to partition $I_n$ into exactly $k$ sets. They obey the recurrence:
      \begin{equation*}
        \stirlingsecond{n}{k} = k\stirlingsecond{n-1}{k} + \stirlingsecond{n-1}{k-1}
      \end{equation*}

      A ``closed'' formula for it is:
      \begin{equation*}
        \stirlingsecond{n}{k} = \frac{1}{k!}\sum_{j=0}^k (-1)^{k-j} \binom{k}{j} j^n
      \end{equation*}

    \subsection{Turán's theorem} No graph with $n$ vertices that is $K_{r+1}$-free can have more edges than the Turán graph: A $k$-partite complete graph with sets of size as equal as possible.

    \subsection{Table of trigonometric integrals}
      A list of common and not-so-common trigonometric integrals:

      \ 

      \begin{tabular}{|c|c|}
        \hline 
        $\int \tan x dx$ & $-\ln |\cos x|$ \\ \hline
        $\int \cot x dx$ & $\ln |\sin x|$ \\ \hline
        $\int \sec x dx$ & $\ln |\sec x + \tan x|$ \\ \hline
        $\int \csc x dx$ & $\ln |\csc x - \cot x|$ \\ \hline
        $\int \sec^2 x dx$ & $\tan x$ \\ \hline
        $\int \csc^2 x dx$ & $\cot x$ \\ \hline
        $\int \sin^n x dx$ & $\frac{-\sin^{n-1} x \cos x}{n} + \frac{n-1}{n}\int \sin^{n-2}x dx$ \\ \hline
        $\int \cos^n x dx$ & $\frac{\cos^{n-1} x \sin x}{n} + \frac{n-1}{n}\int \cos^{n-2}x dx$ \\ \hline        
        $\int \arcsin x dx$ & $x \arcsin x + \sqrt{1 - x^2}$ \\ \hline
        $\int \arccos x dx$ & $x \arccos x - \sqrt{1 - x^2}$ \\ \hline
        $\int \arctan x dx$ & $x \arctan x - \frac{1}{2} \ln |1 - x^2|$ \\ \hline
      \end{tabular}

    \subsection{Generating functions}
      A list of generating functions for useful sequences:

      \ 

      \begin{tabular}{|c|c|}
        \hline 
        $(1,1,1,1,1,1,\ldots)$ & $\frac{1}{1-z}$ \\ \hline
        $(1,-1,1,-1,1,-1,\ldots)$ & $\frac{1}{1+z}$ \\ \hline
        $(1,0,1,0,1,0,\ldots)$ & $\frac{1}{1-z^2}$ \\ \hline        
        $(1,0,\ldots,0,1,0,1,0,\ldots,0,1,0,\ldots)$ & $\frac{1}{1-z^2}$ \\ \hline
        $(1,2,3,4,5,6,\ldots)$ & $\frac{1}{(1-z)^2}$ \\ \hline
        $(1,\binom{m+1}{m},\binom{m+2}{m},\binom{m+3}{m},\ldots)$ & $\frac{1}{(1-z)^{m+1}}$ \\ \hline
        $(1,c,\binom{c+1}{2},\binom{c+2}{3},\ldots)$ & $\frac{1}{(1-z)^c}$ \\ \hline    
        $(1,c,c^2, c^3, \ldots)$ & $\frac{1}{1-cz}$ \\ \hline    
        $(0,1,\frac{1}{2},\frac{1}{3},\frac{1}{4},\ldots)$ & $\ln \frac{1}{1-z}$ \\ \hline    
      \end{tabular}      

      \ 

      A neat manipulation trick is:
      \begin{equation*}
        \frac{1}{1-z}G(z) = \sum_{n}\sum_{k\leq n}g_kz^n
      \end{equation*}
  \end{multicols}
\end{document}
